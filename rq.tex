\startchapter{Research Questions}
Software engineering as other engineering disciplines is concerned with making the process of developing (or engineering) software more efficient.
As software development is a human intensive task, reducing the time spend on any given piece of software, either creating or maintaining it, directly reduces costs.
Both creation and maintenance are both substantially influenced by quality issues discovered by testing the software or reported by customers.
In either case, creation or maintenance, getting the software into a state it can be shipped to the customer is the next important goal of any team.
In this thesis we are focusing on improving upon quality with a focus on improving the success of creating a sufficiently working version of product commonly referred to as build.

The concept of Socio-Technical congruence introduced by Conway~\cite{} especially the refined version by Cataldo et al~\cite{} offers a good starting point to explore why software builds might fail.
Socio-Technical congruence is composed of both the technical relations within a piece of software and the corresponding social relations between developer within a project.
This concept both builds upon existing work of technical as well as social factors that relate to software quality.

In the first half of this thesis we will discuss the impact on social, technical and socio-technical networks on build success.
To understand if it makes sense to look at the concept of socio-technical congruence, which was first conceived to relate to overall productivity instead of software quality, we first need to established if the separate parts of socio-technical congruence namely technical and social factors influence build success, our software quality metric of choice.

Since the end goal is to establish a measure that can be made actionable it is important that at least one of the two sides of the socio-technical concept, social or technical, has an effect on software builds.
Unless this effect exists it will be difficult to manipulate the socio-technical networks to improve the success of software builds.
Therefore we formulate the research questions guiding the first half of this thesis (Chapters~\ref{}-\ref{}) as follows:

\begin{description}
% 
\item[RQ 1:] Does Socio-Technical Congruence influence build success?
  %
  \begin{description}
  \item[RQ 1.1:] Do Technical Networks influence build success?
  \item[RQ 1.2:] Do Social Networks from repositories influence build success?
  \item[RQ 1.3:] Does Socio-Technical Networks influence build success?
  \end{description}
%
\end{description}
% rq 1.1
Research Question 1.1 explores through a literature review in Chapter~\ref{} how technical relationships or the relationships between the units of work produced by a developer influence software quality.
This is important to know since the actual errors are a result of implementation errors that violate explicit dependencies, such as the wrong order of calling methods, or implicit dependencies such as assumptions made about the input of a method. 
Without an influence of technical dependencies among work units and their influence on software quality the concept of socio-technical congruence could not yield much in the sense of explanatory power due to the missing relation to the actual error.

%rq 1.2
Research Question 1.2 deals with the other side of the socio-technical congruence coin namely the social or more accurately communication between developers.
Similarly with the technical dependency, for socio-technical congruence to show an effect that is actually help take corrective measure it will be to bring the right people to talk to each other to sort out errors. 
For this to be possible the actual communication of developers or in the sprit of socio-technical congruence the social network defined by the communication among developers need to have an effect on software quality.

%rq 1.3
With Research Question 1.3 both sides of the socio-technical coin are unified and we explore the effects of socio-technical networks on software quality.
Unless the combination holds predictive power with respect to software quality is can hardly serve as a method to generate recommendations to elevate or prevent issues with software quality.
With this and the previous research questions answered in a positive manner we proceed to the next set of questions that investigate to what extend the concept of socio-technical congruence lends itself to generate useful recommendations.

In the second part of this thesis we will lie the ground work for a recommender system based upon the concept of socio-technical congruence.
We approach the question of whether socio-technical congruence lends itself to make useful recommendations in four steps: (1) Can we extract recommendations after the fact that point to issues in software quality, (2) would developer generally listen to the level of recommendation and (3) when would they do so, and (4) can we generate recommendations at the right time.
From those guiding principles we formulated the following research questions:

\begin{description}
%
\item[RQ 2:] Can Socio-Technical Networks be used to recommend improvements? 
  %
  \begin{description}
  \item[RQ 2.1:] Can Socio-Technical-Networks be used to suggest improvements that matter?
  \item[RQ 2.2:] Does recommending on the lowest level make sense?
  \item[RQ 2.3:] When does recommending on the lowest level make sense?
  \item[RQ 2.4:] Can recommendation be given in a timely manner?
  \end{description}
\end{description}
% rq 2.1
We start the second part of this thesis by exploring if after the fact recommendations can be made using the concept of socio-technical congruence (RQ 2.1 addressed in Chapter~\ref{}).
This needs to hold before it makes sense to continue exploring the feasibility of a recommender system.
% rq 2.2 % rq 2.3
In the following two research questions RQ 2.2 and 2.3 (Chapter~\ref{}) we explore to what extend those recommendations we generate from socio-technical congruence are acceptable to developers in both the level of detail and the timeliness of the recommendations.
% rq 2.4
We will close our investigation with RQ 2.4 exploring the ability to generate those recommendations we extracted post-mortem from the concept of socio-technical congruence in a more timely fashion such that developer can intervene and fix issues sooner or enable them to prevent introducing those issues in the first place (Chapter~\ref{}). 

