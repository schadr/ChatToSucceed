% !TEX root = thesis.tex
\startchapter{Discussion}
\label{chap:disc}
In this chapter we discuss the approach and how the research we presented in the last three chapters support it (Section~\ref{ch:dis:app}).
Following, we go over some general threats to validity.
Then, we shortly describe how we see the implications of making such a recommender system available in Section~\ref{sec:implications} and close with concluding remarks and some future work (Section~\ref{ch:dis:con}).

\section{Revisiting the Approach}
\label{ch:dis:app}
The approach we presented in Chapter~\ref{chap:approach} consist of five steps:

\begin{enumerate}
\item Define scope of interest.
\item Define outcome metric.
\item Build social networks according to the scope in real time.
\item Build technical networks according to the scope in real time.
\item Generate actionable insights.
\end{enumerate}

In Chapter~\ref{chap:soc-net} we showed that communication structure of a software development team influence build success, suggesting that there is value in manipulating this structure to improve the likelihood for a successful build.
That evidence was added to in Chapter~\ref{chap:stc-net2} where we showed that gaps in the social network constructed from developer communication as suggested by technical dependencies among developers also affect build success.

In those two studies we already applied the first four steps of the approach presented in Chapter~\ref{chap:approach}.
We defined the build as the scope together with the build outcome, success or failure, as the outcome metric that we aim at optimizing.
Using the scope we constructed social networks from the communication among developer that can be related to a build as described in Chapter~\ref{chap:bg} in both Chapter~\ref{chap:stc-net2} and~\ref{chap:stc-net}.
Chapter~\ref{chap:stc-net2} used dependencies among change-set committed by developers that are relevant to a given build to construct a technical network to complement the social network forming a socio-technical network.

The three studies we presented in Part~\ref{part3} of this thesis we focused on the last step to \emph{generate actionable insights}.
The study in Chapter~\ref{chap:stc-net} showed that we are able to produce recommendations from available repository data that statistically affect build success.
These recommendations take the form of highlighting two developers that have a technical dependency but did not communicate in the context of the build.
These recommendations in the form of recommending developer to communicate would directly change the structure of the social network, which we intend in order to improve build success.

We decided to focus on generating recommendation enticing developer to communicate in order improve build success over recommendation that would suggest code changes changing the dependencies among developer for two reasons:
(1) proper code changes are more difficult to suggest without a sufficient understanding of the program requiring more in-depth program analysis and 
(2) developer need to trust the recommendation, which is easier to achieve by limiting ourselves to suggesting people that are affected by a change.

In Chapter~\ref{chap:talk} we explored the developer view with respect to recommendation systems and if and when recommendation on a change-set level would be appropriate.
The feedback we received generally welcomed such recommendations as long as they are not seen as irrelevant, corroborating Murphy and Murphy-Hill's~\cite{murphy:rsse:2010} point.
Developers, in fact, discuss change-set in general, but specifically towards the end of a release cycle, as each change becomes more important with respect to the stability of the overall project.

Through an in-class study with a several students in Canada and Finland we investigated whether we can collect the necessary data to compute recommendations at the appropriate time as well as if those recommendation actually could prevent builds from failing (Chapter~\ref{chap:actionable}).
We recognize that a student project is substantially smaller in size and complexity than and industrial project such as Rational Team Concert, nevertheless, the student project was to work on an existing open source project that is used by several companies.
Furthermore, the in class study gave us the chance to see the possible effect of our recommendations more clearly as the smaller complexity allowed build to fail for less reasons and therefore making the impact of the recommendations clearer.

Overall, we gave evidence to the usefulness of our approach by selecting a specific scope and outcome metric, builds and build success respectively, and defined the construction of the social and technical networks in detail (see Chapter~\ref{chap:bg}).
Furthermore, we showed that the approach can generate actionable insights that are acceptable to developer.
In a final study involving students from two countries we found evidence that we are both able to generate recommendation early enough to be acted upon as well as demonstrated that such recommendation could actually prevent build failures.


\section{Threats to Validity}
Although we discussed threats to validity in each chapter that presented research that we conducted in the course of this thesis we  address the overall threat to the approach of leveraging socio-technical networks to generate actionable knowledge as described in the previous sections.

% limited number of studies
\paragraph{External Validity}
In this thesis we the studies we presented draw information from observational studies and studies relying on the same development repository.
Although this limits the generalizability of the findings presented as well as the validity of the inferred approach, we think that the approach still holds merit as the studies that lay the foundation for the validity of generating insights in real time are derived from and industrial project comprising more than one hundred developers at a large software corporation.
This in-depth relationship created by working together with the IBM Rational Team Concert development team limits the amount of data available for the studies we presented but this in-depth relationship enables us to better interpret the collected data as well as gain a deeper understanding of the organization and their processes and how they influence the data.
In the case of the in class study, we aimed to minimize the conclusions we drew to only serve as a feasibility study to demonstrate that technical networks can be constructed in real time as well as give some evidence that potential recommendations could have prevented build failures from occurring.

\paragraph{Construct Validity}
In this thesis we conceptualized social dependencies among developers using digitally recorded communication artifacts in the form of work item discussions as well as relied on technical dependences inferred from developers changing the same source code file.
Both constructs are used by the software engineering research community in several studies.
Nevertheless, both the social and technical dependency characterizations come with the danger that they do not necessary measure social or technical dependencies of relevance or might as well miss existing dependencies.
This leads to the threat that our inferences made based on the data might be due to inconsistencies such as meaningless communication among developers or file changes that are not technical in nature.
Given that we use data that was generate by highly disciplined professionals or by students that we monitored, and thus were able to intervene in case they strayed from following the development process properly, we are confident that the data available for analysis is of high quality.

\paragraph{Internal Validity}
% never traced found patterns to issues/rellied on statistical analysis
Chapters~\ref{chap:stc-net2} and~\ref{chap:stc-net} demonstrated that constructing the socio-technical networks is feasible and in Chapter~\ref{chap:stc-net} we showed that there is a statistical relationship between the network configuration and build success that can be used to generate recommendations.
One issue that we will need to address in future work is showing a definite link between the insights presented in Chapter~\ref{chap:stc-net} and the actual build failures and to what extend the recommendations actually can prevent build failures from happening.
To mitigate this threat we showed some initial evidence of tracing a failed build back to its original failure source and showed that the failure could have been prevented with the socio-technical information available at the point in time when the error was introduced into the code base.

% approach not tested in the field
The final threat to the approach, which is related to the previously mentioned lack of tracing the basis of the recommendations back to actual build failures, is that we did not test it in the field to see how the recommendation affect the development process.
We presented in Chapter~\ref{chap:talk} a study that explored if the recommendations are made at an appropriate level of granularity as well as feedback to the usefulness of such recommendations.
Furthermore, the study conducted in a class room setting also suggests that there is value in generating such recommendations.


\subsection{icse - Threats to validity}
We conceptualized communication based on comments on work items. Besides
that, the Jazz team communicates via email, chat, web-based information and
face-to-face meetings. Based on our observations and conversations with the Jazz
team, we are certain that comments are mostly used to communicate about work
items. Since they are work item-specific and immediately available.

For the network construction, we assumed that every developer commenting
on or subscribed to a work item reads all comments of that work item. This
assumption might not always be correct. By manual inspection of a selected number
of work items, we found that developers who commented on a work item are aware of
the other comments, confirming our assumption.

Due to storage problems the Jazz teams erased some build results. In the case of
nightly builds we expected 90 builds (according to project duration) but found
only 15. This might affect our results but we argue that due to our richness of
data the general trend is still preserved. Hence we expect that our findings
would be supported by the complete data, too.

Our results are only valid for the Jazz project and can not be generalized to
any software development project. Additional studies on different projects
using Jazz could be conducted to confirm our results.

\subsection{tse - Threats to Validity}
\label{sec:threats}
Socio-technical congruence can be difficult to compare between studies. Conceptualizations, such as technical dependencies and actual coordination, vary from project to project. Some project-specific conceptualizations we used are the build as a measure of success, the fact that only files that are touched by more than one individual qualified as having dependencies on each other, and the work item comments connecting people in a clique.
Because of the context-sensitive nature of socio-technical congruence, especially with respect to the construction of communication and dependencies, it is difficult to apply socio-technical congruence as a benchmark for coordination. Having an understanding of the context of the project is extremely important when interpreting results obtained from socio-technical congruence calculations.

Our coordination needs are likely overestimated due to the modular nature of RTC.
As we are studying a distributed project that follows a transparent development process, RTC uses a modular design \cite{maccormack2006} and a change in a file in a change set may not necessarily be dependent on other changes within the same file. 
We observed evidence in our study that the dependencies among change sets as well as the attached work items may not be as strong as we have originally believed.

Our conceptualization of actual coordination underestimates the amount of communication that truly occurs in the project. 
We relied on repository data in RTC and were unable to conceptualize forms of communication such as instant messenger, phone calls, and face-to-face interactions.
Due to geographical distance and time zone differences, the RTC team's primary mode of collaboration is the work item comment system, but we made a number of assumptions about commenting behaviour.

First, we assume that everyone involved in the work item reads every comment. Second, we do not take into consideration any additional coordination that may occur from a silent onlooker reading the comments. We believe intuitively that the first effect is greater than the second.

Another threat to validity is that our data does not cover every build executed in the lifetime of RTC. RTC does not keep a full archive of build results, and as a consequence, we do not have a full population from which to draw data from. The threats are particularly high for early data points. The large confidence intervals in some of the builds, namely nightly builds, reflect this lack of data, thus we hesitate to draw conclusions based on early builds and nightly builds.

Finally, this study is a single case study and the results are not generalizable, nor can they be directly compared to existing studies due to the differences in the projects under examination. However, we believe that this study advances the theoretical and the empirical examination of socio-technical congruence, and raises a number of questions that are worthwhile for future study.

\subsection{fse -Threats To Validity}
\label{ch8:threat}
During our study we identified two main threats. 
One threat covers issues that arise from the underlying data we used.
The other threat deals with possible problems from the conceptualization of
constructs in our study.

\subsubsection{Data}
We performed all our analysis on one set of data, the Rational Team Concert
repository. This limits the generalizability of our findings, due to the fact
that we only made the observations within one project. The  
project size and the project properties -- incorporating open
source practices such as open development and encouraging community involvement
-- make us believe that our findings still hold value.

Furthermore we only investigated three months of the project's lifetime. This
might lead to smaller significance of our results. However, since the three
months are directly before a major release of the project, this dataset contains
the most viable data for our analysis. In those three months a lack of necessary
coordination is the most harmful to the project.

Another threat that is inevitable in studies of software engineers is
the possible lack of recorded communication. This and the possibility of people
coordinating without communicating, such as reading each others source
code~\cite{bolici:stc:2009}, are mitigated in Rational Team Concert by its development process.
The Rational Team Concert team's development process demands that the developers
coordinate using workitem discussion. 


\subsubsection{Conceptualization}
Our conceptualization of the three edges we use to construct the
socio-technical networks might introduce inaccuracy in our findings.
First, social edges are extracted from workitem discussions. We assumed that
every developer commenting on or subscribed to a work item reads all comments of that
work item. This assumption might not always be correct. By manual inspection of
a selected number of work items, however, we found that developers who
commented on a work item are aware of the other comments, confirming our assumption.

Second, the technical edges are not problematic by themselves, but they are not
complete, since there are more technical relationships between developers that
can be examined. For example, two developers can be connected if one developer
changes someone else's code. This however does not invalidate our findings, it
just suggests that there is room for improvement, which we should address
next.

Third, socio-technical edges on the other hand may suffer from the combination of
social and technical edges. For example, it is not necessarily true that the
discussion of  two developers in a technical dependency is always about their
technical dependency. 
In our study however, since the changes to
source code files we use to extract technical dependencies are attached to workitem discussion, we are
confident that they addressed the changes at least indirectly.






\section{Contributions through Empirical Studies}
\section{fse - Practical Implications}
\label{sec:implications}
Our findings have several implications for the design of collaborative systems.
By automating the analyses presented here we can incorporate the knowledge about
developer pairs that tend to be failure related in a real-time recommender
system. Not only do we provide the recommendations that matter to the upcoming
build, we also provide incentives to motivate developers to talk about their
technical dependencies. 
%
Such a recommender system can use project historical data to
calculate the likelihood that an upcoming build fails given a particular
developer pair that worked on that build without communicating to each other.

For management, such a recommender system can provide details about the
individual developers in, and properties of, these potentially problematic
developer pairs. Individual developers may be an explanation for the behaviour of
the pairs we found in Rational Team Concert. This may indicate developers that are
harder to work with or too busy to coordinate appropriately, prompting management
to reorganize teams and workloads. This would minimize the likelihood of a build
to fail, by removing the underlying cause of a pair to be failure related.
Similarly, as another example from our study, most developer pairs
consisted of developers that were part of different teams. In such
situations management may decide to investigate reasons for coordination
problems that include factors such as geographical or functional distance in the project.

\subsection{icse - Practical implications}
\label{subsec:practicalimpl}
Our model can be used by Jazz teams to assess the quality of their current
communication in relation to the result of their upcoming integration. If a team
is currently working on a component and an integration build is planned in the
near future, the measures of the current communication in the team can be
provided as input to our prediction model and the model will predict whether the
build will fail with a precision shown in Table~\ref{tab:PredictionResultTable}.
For example, if team P is working towards a build and our model predicts that the
structure of its current communication leads to a failed build, the team can have
a 76\% (see Table~\ref{tab:PredictionResultTable}) confidence that the build is
going to fail. This information can be used by developers in monitoring their
team communication behavior, or by management in decisions with respect to
adjusting collaborative tools or processes towards improving the integration.

In our analysis we used the complete communication of the build related time
interval as input for the model to predict the build result. In
Figure~\ref{fig:ReducedCommunicationInput}, this is the time interval from $t_0$
to $t_1$ for $Build~N$. This time interval might not be appropriate for practical
applications in which the build process takes only a view minutes, as $t_1$ is
immediately before the build starts. Predicting the build result has no value for
the developers when they have no time left to react on a build failure prediction
and they can simply run the build instead of predicting the result. For long
builds that last several days, the prediction at $t_1$ is still practical
relevant. Developers might delay the build to perform reactive changes on the
failure prediction.

In an additional analysis we investigated the prediction recall and precision by
only considering the communication of reduced time intervals as input for our
model. For each build N result, we trained our model with the complete
communication of all build results except the one from N, and predicted the
result N by only considering the first 25\%, 50\% and 75\% of the communication
as input for the prediction model (see
Figure~\ref{fig:ReducedCommunicationInput}).

We only show the precision and recall of the prediction for the 25\% time range
in Table~\ref{tab:Prediction25PResultTable}, as they are already high enough for
practical use. We observe that the model has almost the same predictive power as
the model considering the complete time range of the build. A possible
explanation is the involvement of work items in many builds. That means that the
task completion time of a work item is longer than the time interval of a build
and task related change sets and communication occurs in multiple builds. When a
developer delivers a change set in the first 25\% of the build related time
interval and the change set is associated to a work item that has already been
discussed, the complete past communication is included. Including the past
communication is valid, as our prediction model learns from the communication
history. In Jazz, a work item is in average included in 5.17 builds (min=1,
median=3, max=57), which provides a possible explanation for having valuable
prediction results even early in the build related time interval.

With the results of this analysis, we are confident that our prediction model is
practically useful, even if it does not consider the complete communication of a
build. Thus, developers and managers have time to react on build failure
predictions before the actual build fails.



\subsection{Conclusions and Future Work}
\label{ch:dis:con}
In this thesis we illustrated an approach to leverage the concept of social-technical congruence to generate actionable knowledge.
This five step approach focuses on defining two key parameters up front: (1) the scope of interest and (2) the outcome metric of interest.
The first parameter scope helps with constructing the social networks (the third step) and constructing the technical networks (the fourth step) by supporting the selection of the best data sources.
The outcome metric guides the analysis to produce actionable knowledge in the form of indicators that positively or negatively influence the outcome metric (step 5). 

We derived this approach through two case studies that investigate the usefulness of social and socio-technical networks with respect to supporting developers improve software quality (Chapters~\ref{chap:soc-net} and~\ref{chap:stc-net2}).
We detailed the approach in Chapter~\ref{chap:approach} and conducted a study to see if the approach can generate statistically relevant recommendations in Chapter~\ref{chap:sec-net}.
The studies we conducted in the subsequent Chapters~\ref{chap:talk} and~\ref{chap:actionable} further explores the usefulness of the information with respect to whether experts expect the level of recommendations to be of use as well as if these recommendations could be produced in real time and potentially prevent issues from arising.

Each study by itself contributed the the overall body of knowledge besides furthering the goal of the thesis to lay the foundations of a recommendations system leveraging the construct of socio-technical congruence.
With our first study we gave empirical evidence of communication among software developers influencing software quality.
Although by itself not surprising that issues in communication can hinder productivity and introduce ambiguities that might lead to problems with respect to software quality, it is, to the best of our knowledge, the first study that instead of looking into content of individual  conversations takes a higher level approach and relates communication structures to software quality.

% stc and build success
The relationship between communication structure and build failures however significant has only a small effect on the overall success rate of software builds, the outcome metric we studied.
This lead us to include information about the system by adding technical dependencies as expressed by the source code among software developers.
Backed up by findings in the research area of socio-technical congruence we hypothesized that the technical relationships help to zero in onto the important relationships among developers that relate to build failures.
As the relationship between socio-technical congruence and productivity suggested influence on software quality, we showed in Chapter~\ref{chap:soc-net} that it actually predicts build failures with varying accuracy depending on the type of build.

% failure inducing pairs
Being able to predict whether a build fails already helps developer to plan ahead with respect to future work, such as stabilizing the system in contrast to working on new features, but ultimately we want to be able to prevent builds from failing.
To that purpose we would need to influence the socio-technical network such that it takes a structure that is more favourable to build success.
We found that certain constellations within a socio-technical network, to be more precise pairings of software developer and their respective relationship, seem to be correlating with build success.
This evidence can be used to recommend action before the build is commenced in the sense that developer can investigate their relationship by for example discussing the code changes that created a technical relationship between them.

% talk or not to talk
Recommendations to manipulate the socio-technical network to improve the odds of a successful build is a good start, however the developers that need to follow these recommendations need to accepts them first.
It turns out, that developers are generally open to recommendations on a low level, such as on a change-set basis, but it depends on external factors such as the development process.
For instance, we found depending on how close a development team to a software release is the more they focus on the implications of individual changes, whereas developers focus more high level reusability issues at the beginning of a release cycle.

% leveraging stc in real time
Knowing that socio-technical congruence lends itself to produce actionable knowledge that has an acceptable form to support developers in the wild.
The final peace detailed in Chapter~\ref{chap:actionable} was to show the feasibility of generating the recommendations at the right time.
Thus, we showed that socio-technical congruence can be used in real time to create actionable knowledge that might be of use to developers.

% future work
The work presented in this thesis lend it self to several obvious venues of future work, such as building and testing the recommendation system with several software development teams to study its impact.
A more interesting avenue to pursue is to explore what software architecture can support what kind of communication and organizational structure.
So far, the research around socio-technical congruence is pointing into the direction of changing how software developers coordinate their work, but we propose to return to the original observation Conway made in that the software architecture will change to accommodate the communication structures inhering in an organization.
Therefore, analyzing software architectures with respect to the project properties, such as distribution of the development team or the organizational hierarchy, might yield valuable insight in guiding design decisions of the software product that not only take into account properties to increase the feature richness or maintainability of the software product but is optimal with respect properties of the organization and the development team in order to increase productivity and quality. 
