% !TEX root = ../thesis.tex
\newpage
\TOCadd{Abstract}

\noindent \textbf{Supervisory Committee}
\tpbreak
\panel

\begin{center}
\textbf{ABSTRACT}
\end{center}
Efficient coordination among software developers is a key aspect in producing high quality software on time and on budget.
Several factors, such as team distribution and the structure of the organization developing the software, can increase the level of coordination difficulty.
However, the problem runs deeper as it is often unclear which developers should coordinate their work.

In this dissertation, we propose leveraging the concept of socio-technical congruence (which contrasts coordination needs with actual coordination) to improve the social interactions among developers 
by devising an approach and its implementation into a recommender system that identifies relevant coordinators.
Our unit of analysis is the integration build, whose outcome represents the quality of coordination.
After development, this approach was applied in a number of IBM Rational Team Concert development team case studies, as well as a large student project at the University of Victoria, Canada, and Aalto University, Finland.

Since each software product is just the latest integration build, it is of utmost importance for the industry to ensure a failure free build.
While developing an approach to improve coordination among software developers we uncovered that unmet coordination needs, as well as the communication structure in a team, have a significant influence on build outcome.