\newpage
\TOCadd{Abstract}

\noindent \textbf{Supervisory Committee}
\tpbreak
\panel

\begin{center}
\textbf{ABSTRACT}
\end{center}
Efficient coordination among software developer is on of the key aspect to producing high quality software on time and on budget.
Many factors such as a distributed team or the structure of the organization developing the software can increase the level of difficulty to coordinate smoothly.
But the problem runs deeper, because often it is unclear which developers should coordinate their work.

The concept of socio-technical congruence, which matches actual coorindation among a development team with the coordination needs represented by source code dependencies among source code software developers changed, shows a correlation with productivity~\cite{}.
More interestingly, the concept of socio-technical congruence suggests areas where developers can improve their coorindationi, namely those areas with unfulfilled coordination needs.
One key limitation of socio-technical congruence is its dependence on data that is only available after the work has been accomplished, thus making it difficult to use the insights socio-technical congruence for preventive measures.

Therefore, in this thesis we leverage the concept of socio-technical congruence to generate actionable knowledge, we demonstrate two ways to generate actionable knowledge: (1) shifting the focus from increasing productivity to that of software quality, where providing recommendation after a developer completed a task to remidee residual issues is acceptable and (2) by building on work presented by Blincoe et al~\cite{} to gather information through tools such as Mylyn~\cite{} to approximate coordination needs in real time.

We syntehsized our findings, that there exists patterns that significantly and substantially correlate with build failures, in a step wise approach to extract actionable knowledge that helps in building socio-technical networks as well as generating actionable knowldge from them.
