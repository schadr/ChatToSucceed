\startchapter{Meet IBM Rational Team Concert}
In this chapter we will describe the IBM Rational Team Concert\footnote{https://www.jazz.net} product as well as its development team.
Since most of the research carried out throughout this thesis was in collaboration with this product team which as part of their development process is also using their own product.
We will begin with introducing the product and its functionalities before we describe the development team's composition and development process.

\section{The IBM Rational Team Concert Product}
IBM Rational Team Concert (RTC) is a server client application mean to support software development.
It comprises four major functionalities that bear relevance to this thesis:
(1) source control management that is based on similar principles as git\footnote{} and mercurial\footnote{}, (2) work item management to keep track of issues, tasks and features that need to be resolved and implemented, (3) planning capabilities to organize work items into iterations working towards milestones, and (4) a build engine that allows for continues builds and regression tests.

\subsection{Source Control}
The source control system provided by Rational Team Concert is server based similarly to traditional version control systems such as CVS\footnote{} and SVN\footnote{}.
Yet, each developer has their own workspaces similarly to local git or mercurial repositories, which a developer is working on.
These branches or in RTC terminology streams can contribute their changes, recorded in change-sets to other streams shared by teams or can be exchanged between workspaces.

With this the main difference between the RTC source code management system and a distributed source code management systems such as GIT lies with all the repositories or streams being present on one single machine as opposed to many different machines.
Even branching and sharing of changes to the source code work similarly to distributed version control systems as they can be freely moved in-between workspaces, in-between streams, and between streams and work spaces.

\subsection{Work Items}
The work items component serves two functions: (1) it allows for customer interaction as it allows a customer or user to file feature requests and file bug reports and (2) it allows developers to additionally to manage their work by creating tasks as well as discuss and document thought about a defined unit of work.
These work items are basically an extension to bug reports as filed to a bugzilla instance\footnote{}, since they serve more purposes than capturing bug reports and feature requests.

Developer have the ability to link work items into logical units and build hierarchies of work items to structure their work.
Furthermore, work items allow tracking meta information such as time estimates, severity and priority and on top of that lends itself to be customized with self defined fields on a project configuration level.


\subsection{Planning}
In contrast to issue trackers such as bugzilla IBM Rational Team Concert offers planning capabilities.
Within RTC plans can be defined that consists of a number of work items that can be ranked against each other and assigned to different team members.
RTC also supports multiple plans for different teams to support a better overview for the teams instead of cluttering plans with less relevant tasks.

The planning component supports several development styles reaching from agile-development styles such as XP~\cite{} and SCRUM~\cite{} to more traditional development methods such as Waterfall~\cite{}.
Each different development style comes with a different default of control measures and charts such as burn down charts.
These charts as well as the planning can be both reviewed in the Eclipse and web client.
Both can be viewed in a customizable dashboard for quick information retrieval to stay up to date.

\subsection{Build Engine}
The RTC build engine is comparable to build engines such as IBM Rational Buildforge\footnote{http://www-01.ibm.com/software/awdtools/buildforge/enterprise/} and Jenkins\footnote{http://jenkins-ci.org/}.
As build engine it allows for both, automated builds in pre-defined intervals such as nightly builds as well as on demand builds from users.

\subsection{Foundation/Integration}
The IBM Jazz Foundation builds the layer on which the previously described components run.
By itself it does not offer any capabilities to the user, but it enables each component to integrated with the others. 
This allows the developer to both manually link artifacts with each other and define queries that cross reference multiple repositories.

For example, a developer is working on implementing a work item and create changes to the code base that he bundles into a change-set.
Before she submits the change-set resolving the work items she links that change-set to the work item.
This enables her colleagues to easily accept the change-set into their  workspaces to both review and test them.
Once the work item is resolved, the plan and the burndown charts update to reflect this change visible to anyone with access to the plan and charts.

Developers often need to modify existing code and therefore need to understand the thought process that went into the code they are looking at~\cite{}.
For this purpose developers can create queries that look for work items associated with changes to this code parts and instead of reading source code, they have access to documentation related to the code they are investigating.


\section{The IBM Rational Team Concert Product Development Team}
The IBM Rational Team Concert product development team is distributed across multiple sites following a agile development methodology.
In this section we are going to characterize the development team at large as well as describing their development process.

\subsection{The People}
The Jazz team is a large distributed team and uses the Jazz platform for development. 
The Jazz development involves distributed collaboration over 16 different sites located in the United States, Canada, and Europe. 
Seven sites are active in Jazz development and testing. 
There are 151 active contributors working in 47 teams at these locations, where contributors belong to multiple teams. 
Each team is responsible for developing a subsystem or component of Jazz.
The team size ranges from 1 to 20 and has an average of 5.7 members. 
The number of developers per geographical site ranges from 7 to 24 and is 14.8 in average.

\subsection{The Process}
The project uses the Eclipse Way development process~\cite{frost:ieeesoftware:2007}.
It defines six-week iteration cycles, which are separated into planning,
development and stabilization activities. A project management committee
formulates the goals and features for each release at the beginning of the each
iteration, and Work Item's represent assignable and traceable tasks for each
team.

\pagebreak
\section{Copied from all over}
\subsection{Coordination and integration in Jazz}




\begin{figure}[t]
\begin{center}
\includegraphics[width=8.3cm]{figures/BuildResult}
\caption{Teams contribute to their own source streams, which are then merged into one project stream.}
\label{fig:BuildResult}
\end{center}
\end{figure}


As illustrated in Figure~\ref{fig:BuildResult}, the coordination process within
each iteration requires the integration of subsystems developed by individual
Jazz teams in a major milestone build of the product (referred to as beta build).
Each team owns a source code \et{Stream} for collaboration and concurrent
implementation of the subsystem. A Stream is the Jazz equivalent to a branch of a
source configuration management system such as Subversion.

A continuous integration process takes place at team-level or project-level. In
frequent intervals, each integration build (referred to a build henceforth)
compiles, packages, and tests the source code of a stream. At the team-level,
contributors commit code changes that are encapsulated in \et{Change Sets} from
their own workspace to the \et{Team Stream}. The team integrations build the
subsystem developed by the team. Once a team has a stable version within the Team
Stream, the team publishes the change sets into the Jazz \et{Project Integration
Stream} (see Figure~\ref{fig:BuildResult}). At the project level, the automated
Jazz integration builds the subsystems of all teams. The Jazz project-level
integration takes place \et{nightly}, \et{weekly}, and at the end of each
iteration -- \et{beta} build.

\subsection{Coordination outcome measure}
In our study we conceptualize the coordination outcome by the \et{Build Result},
which is regarded as a coordination success indicator in Jazz and can be \error,
\texttt{WARNING} or \ok. We analyze build results to examine the integration
outcomes in relation to the communication necessary for the coordination of the
build.

Conceptually, the \texttt{WARNING} and \ok\ build results are treated similar by
the Jazz team, as they require no further attention or reaction from the
developers. In contrast, \error\ build results indicate serious problems such as
compile errors or test failures and require further coordination, communication
and development effort. We thus treated all \texttt{WARNING}s as \ok s to clearly
separate between failed and successful builds in our conceptualization of
coordination outcome.


\subsection{Communication in Jazz}

\et{Commenting} on work items is the main task-related communication and
collaboration channel used by Jazz. \et{Work Item}s represent single assignable
and traceable tasks. A set of work items defines the tasks for each team in each
iteration. Different types of work items represent defects, enhancements, and
general tasks. The work items are assigned to and owned by contributors. The
coordination necessary around the implementation of a work item is facilitated by
contributors commenting or observing the communication around work items.









%%%%%%%%%%%%%%%%%%%%%%%%%%%%%%%%%%












\subsection{Development and Builds in the Jazz Team}
The Jazz\tm\ team is a large distributed team and uses the Jazz\tm\ platform for
development. The Jazz\tm\ development involves distributed collaboration over 16
different sites located in the United States, Canada, and Europe. Seven sites are
active in Jazz development and testing. There are 151 active contributors
working in 47 teams at these locations, where developers can belong to multiple
teams. Each team is responsible for developing a subsystem or component of Jazz\tm\.
The team size ranges from 1 to 20 and has an average of 5.7 members. The number
of developers per geographical site ranges from 7 to 24 and is 14.8 in average.

The project uses the \emph{Eclipse Way} development process~\cite{frost:ieeesoftware:2007}.
It defines six-week iteration cycles, which are separated into planning,
development and stabilization activities. A project management committee
formulates the goals and features for each release at the beginning of the each
iteration, and \emph{work items} represent assignable and traceable tasks for each
team.
Furthermore, the Jazz\texttrademark\ team's development process demands that the developers coordinate using work item discussion. 

The coordination process within
each iteration requires the integration of subsystems developed by individual
Jazz\tm\ teams in a major milestone build of the product.
Each team owns a source code Stream for collaboration and concurrent
implementation of the subsystem. A Stream is the Jazz\tm\ equivalent to a branch of a
source configuration management system, such as Subversion.

A continuous integration process takes place at team-level or project-level. In
frequent intervals, each integration build (referred to a build henceforth)
compiles, packages, and tests the source code of a stream. At the team-level,
contributors commit code changes that are encapsulated in change sets from
their own workspace to the Team Stream. The team integrations build the
subsystem developed by the team. Once a team has a stable version within the Team
Stream, the team publishes the change sets into the Jazz\tm\ Project Integration
Stream. At the project level, the automated
Jazz\tm\ integration builds the subsystems of all teams. 


We mined the Jazz project repository between April and July 2008, and analyzed a
total of 328 builds, out of which 99 were failed and 227 were successful.
Table~\ref{tab:jazzbuildinfo} contains summary statistics describing the Jazz\tm\
repository. We report different aggregations (minimum, average and maximum) of
number of work items, change sets, and developers over all builds. 






%%%%%%%%%%%%%%%%%%%%%%%%%%%%%%%%%%%%%%%




\subsection{IBM Rational Team Concert Case Study}
IBM Rational Team Concert, or \textbf{RTC}, is a collaborative software development tool built on the Jazz\textsuperscript{\texttrademark} architecture. It incorporates the integrated development environment, the source-code control system, the issue-tracking system, and collaborative development tools. The RTC team self-hosts, meaning that the team uses its own software to develop subsequent versions.

The following qualities make RTC an attractive case in which to study
socio-technical coordination: (1) the distributed team, (2) the iterative
process, and (3) the rich data repository. Because distribution makes coordinating work difficult~\cite{hinds:cscw:2006,herbsleb2003:speed}, RTC is supposed to enhance communication among project members by providing collaborative facilities~\cite{frost:ieeesoftware:2007}. Developers explicitly record communication across sites using the product.
RTC's iterative process encourages developers to ``build early and often''. In addition to supporting continuous builds, RTC does frequent integration builds, providing a large amount of data to analyse.

At the time of our study, the RTC development team involved a team distributed over 16
different sites located in the United States, Canada, and Europe. Seven sites were
active in RTC development and testing. There were 151 active contributors
at these locations. Each team, which was not necessarily confined to one geographical location,
was responsible for developing a component of RTC.
The team sizes ranged from 1 to 20 and had an average of 5.7 members. The number
of developers per geographical site ranged from 7 to 24 and was 14.8 on average.

The project uses the \emph{Eclipse Way} development process~\cite{frost:ieeesoftware:2007}.
It defines six-week iteration cycles, which are separated into planning,
development and stabilization activities. A project management committee
formulates the goals and features for each release at the beginning of the each
iteration, and \emph{work item}s represent assignable and traceable tasks for each
team. RTC's process encourages frequent building, including continuous, nightly, weekly, and integration builds.

One unique aspect of RTC is its ``open-commercial'' development model. RTC is a commercial product developed by IBM, but has a publicly-accessible issue-tracking and reports system. This open-commercial model allows a large amount of information to flow into the team from the community.
Communication between individuals is done through the issue-tracking comments, through instant messaging, through Internet Relay Chat, by phone, and face-to-face. Personal email is discouraged. The team uses a mailing list to deliver announcements, such as server maintenance schedules. 




%%%%%%%%%%%%%%%%%%%%%%%%%%%%%%%%%%%%%%%%%%%%%%%%%%%%%%%%



\subsection{Study Setting}
\label{sec:rtc}
Rational Team Concert is a scalable and extensible team collaboration platform that integrates source code management, issue tracking, and planning tools. The RTC platform uses a client-server architecture, with the server acting as a central data repository.
The team also uses RTC to develop RTC, and in this way RTC is a self-hosting project.
This helps the RTC team become aware of issues with the product before customers.
The project follows a large scale iterative development process that emphasizes transparency~\cite{frost:ieeesoftware:2007}.
It defines six-week iteration cycles, which are  separated into planning, development and stabilization activities. A project management committee formulates the goals and features for each release at the beginning of each iteration, and the output of this process is assigned to the teams in the form of work items. The work items and nearly all documentation are publicly available.

The RTC development team has over 100 developers distributed across multiple sites in North America, Europe and Asia.  The RTC development team is subdivided into several teams that manage a component each, such as the source code system or the web user interface. To communicate with each other, especially across different geographical locations, the developers extensively use RTC's own issue and task management system. A particular characteristic of the RTC development environment---and one that made this study of communication behaviour around changesets possible---are explicit traceable links between work tasks, associated changesets, and developers' online comments about these tasks and changesets\footnote{Changesets in RTC are created in a developer workspace; each changeset consists of several changes to code files.}.
