% !TEX root = thesis.tex
\startchapter{Acceptability of Recommendations}
\label{chap:talk}
Knowing that our approach can generate statistically significant recommendations is only one necessary step towards showing its usefulness.
As Murphy and Murphy-Hill~\cite{murphy:rsse:2010} pointed out recommendations are of little use if developers reject them~\cite{schroeter:cscw:2012}.
Thus in this chapter we pursue the following research question:

\begin{description}
  \item[RQ 2.2:] Do developers accept recommendations based on software changes to increase build success? 
\end{description}

We would like developers to communicate as soon as we suspect that otherwise the likelihood for a build to succeed diminishes.
But currently our investigation only took a very technical stance and ignores many other criteria that might be of relevance to developers.
For instance, it is unlikely that every build is of equal importance.
Furthermore, we also think that developers' opinion of each other plays a role, such as trust and respect.
Therefore, it is important to explore those less technical aspects of developer relationships to gain a better understanding on when to deliver a recommendation.

% give chapter overview
In this chapter, we start by detailing the study design that is relevant to exploring our research question (cf. Section~\ref{sec:studydesign}).
Then, we present the findings we obtained in Section~\ref{sec:findings}.
We conclude this chapter by offering an answer to our research question and leading into the subsequent Chapter~\ref{chap:actionable} (cf. Section~\ref{sec:conclusions}).


\section{Study Design}
\label{sec:studydesign}
We complement our previous analysis of the Rational Team Concert (RTC) development team repositories with rich qualitative data from interviews and observations, considering that much information on a project's lifecycle is not recorded in its repositories~\cite{aranda:icse:2009}. We focused on studying what factors cause a developer to seek information about a change-set, since change-sets are the smallest units of change to a project that directly impact other developers.

Our research question calls for a mixed methods empirical study. Therefore, we collect data about the team's information-seeking behaviour from three different sources.
First, the author of this thesis was embedded in a sub-team of the RTC team as a participant-observer for four months.
Second, we deployed a survey with the entire development team to validate some of the findings from the observations.
Third, we conducted interviews with the developers from one component team to obtain richer information about their communication behaviour.


\subsection{Data Collection}
We used a mixed methods approach to answer our research question. We obtained data from three sources: participant observation, a survey, and a set of interviews with RTC team members.

\subsubsection{Participant-Observation}
We joined one of the RTC development sites in Europe for a four-month period in the Fall of 2010. We were involved as an intern with this development team helping with minor bug fixes, feature development, and testing, and thus was in a position to directly participate in the project development and communication activities. Our data collection opportunities were thus much richer than in typical observations conducted over shorter periods of time, or that do not involve active participation in the project.  During our period as a participant-observers, we kept a daily activity log, recording any information of relevance to the communication behaviour of our team.

The observation period coincided with the months prior to a major release and during which the team focused on extensive testing rather than new feature development. As such, the majority of our observations are concerned with activities around testing the integration of RTC with other IBM Rational products, and it offered the experience of collaborating with many developers from different teams and across the remote sites. Finally, in the one month following the release, we also participated in the development of a feature for the upcoming version of the product.

The team in which we were embedded had the responsibility to develop the task management and the planning components of RTC. There were ten developers at the site, including two novices that had recently joined the team and eight experienced developers. The team includes RTC's three senior project management members, who play a major role in planning the entire product's architecture and development.


\begin{sidewaystable}
\centering
%\subfloat[Process-related items and quotes]{
\small
\centering
\caption{Process-related items and quotes}
\begin{tabular}{l@{\hspace{7pt}}l@{\hspace{-20pt}}r}
\toprule
Survey Items & Interview Quotes (I)/Participant Observation Quotes (O) & Mean Rank\\
\midrule
Release endgame & (O) \emph{``Adrian, in the endgame we only do minimal changes. Is your change minimal?''}& 3.79\\%319\\
To review the change &(I) \emph{``I often lack sufficient understanding of the part of the code I'm reviewing.''}& 5.00\\%311\\
To approve the change &--- & 5.11\\%298\\
Late milestone &--- & 5.14\\%293\\
During iteration endgame &(O) \emph{``Adrian, in the endgame we only do minimal changes. Is your change minimal?''}& 5.42\\%288\\
Obtain a review for the change &(O) \emph{``It is demotivating to get a reject, thus I talk to my reviewer beforehand.''}& 5.55\\%287\\
Obtain an approval for the change &(O) \emph{``I already fixed it, but I  need to convince my team lead to give me approval.''}& 5.72\\%281\\
Related work item has high severity &--- & 6.00\\%246\\
Verify a fix &(I) \emph{``Often I need to ask how I can tell that a change-set actually fixed the bug.''}& 6.37\\%235\\
Topic of work item the change is attached to &--- & 7.64\\%181\\
Priority set by your team lead&--- & 8.55\\%152\\
Role of the committer (e.g. developer) &--- & 9.00\\%145\\
Early in an iteration &(I) \emph{``... there are weeks I sometimes don't talk to my colleagues at all.'''}& 11.24\\%66\\
Early milestone &--- & 12.10\\%60\\%\emph{``''}
\bottomrule
\end{tabular}
%}\vspace{0pt}
\label{tab:sub-process}
%\end{table}
\end{sidewaystable}

%\begin{table}[t]
\begin{sidewaystable}
%\subfloat[Developer-related items and quotes]{
\small
\centering
\caption{Developer-related items and quotes}
\begin{tabular}{l@{\hspace{5pt}}l@{\hspace{-10pt}}r}
\toprule
Survey Items & Interview Quotes (I)/Participant Observation Quotes (O) & Mean Rank\\
\midrule
Author is inexperienced &(O) \emph{``Adrian, please put me always as a reviewer on you change-sets.''}& 2.60\\%385\\
Author recently delivered sub-standard work&(I) \emph{``She just changed teams and still needs to get used to this component.''}& 3.04\\%335\\
Author is not up to date with recent events &(I) \emph{``After you return from vacation we ensure you follow new decisions.''}& 3.15\\%329\\ 
You do not know the change-set author &(I) \emph{``I'd be very irritated if someone other than [...] would touch my code.''}& 5.09\\%269\\	
Author is currently working with you &(O) \emph{``We sometimes discuss changes we made to brag about a cool hack.''}& 5.56\\%226\\	
Author part of same feature team &--- & 6.00\\%190\\ 	
Author part of your team &--- & 7.00\\%188\\ 		
Worked with author before &--- & 7.77\\%165\\ 			
Busyness of yourself &(I) \emph{``If I see a problematic change-set, I'll ask for clarifications even if I'm busy.''}& 8.05\\%164\\
Busyness of the author &(I) {\small\emph{``If I need to know why an author made that change I just contact him.''}}& 8.55\\%147\\
Met in person &(I) \emph{``I've worked with him for 5 years now but never even met him.''}& 9.57\\%117\\
Physical location &(O) \emph{``Here, America or Asia, I don't care, I ping them when they are online.''}& 10.14\\%112\\ 
Author is experienced &--- & 11.00\\%93\\ %\emph{``''}		
Author recently delivered high-quality work &--- & 11.09\\%86\\%\emph{``''}
\bottomrule
\end{tabular}
%}\vspace{0pt}
\label{tab:sub-social}
%\end{table}
\end{sidewaystable}

%\begin{table}[t]
\begin{sidewaystable}
%\subfloat[Code-change-related items and quotes]{
\small
\centering
\caption{Code-change-related items and quotes}
\begin{tabular}{l@{\hspace{-10pt}}l@{\hspace{-20pt}}r}
\toprule
Survey Items & Interview Quotes (I)/Participant Observation Quotes (O) & Mean Rank\\
\midrule
Changed API &(O) \emph{``Adrian, why did you change that API?''} [The RTC team avoids changing API.] & 2.84\\%390\\            	
Don't know why code was changed &(O) \emph{``I don't see a reason why you changed that code, you sure you needed to?''}& 3.00\\%348\\ 
Affects frequently used features &(I) \emph{``Before I ok a fix [even to a main feature], I ask if there is a workaround.''} & 4.10\\%325\\ 
Complex code &--- & 4.89\\%289\\	%\emph{``''}
Introduced new functionality &--- & 6.14\\%260\\ %\emph{``''}
Is used by many other methods &(O) {\small\emph{``Why did you fix this part? The bug is not in this part of the code,}}& 6.41\\%240\\ 
&\emph{ it is used and tested a lot.''}&\\
Your code was changed &(I) \emph{``I'd be very irritated if someone other than [...] would touch my code.''}& 6.42\\%232\\ 
Stable code was changed &(O) {\small\emph{``Why did you fix this part? The bug is not in this part of the code, }}& 6.62\\%213\\
&\emph{it is used and tested a lot.''}&\\
Change unlocks previously unused code &--- & 7.42\\%192\\ 
A bug fix &--- & 8.61\\%162\\%\emph{``''}	
A re-factoring &(O) \emph{``I separate a re-factoring from a fix so that people can }& 9.15\\%152\\
&\emph{ask me questions about the fix.''}&\\
Frequently modified code was changed &(I) \emph{``I like to know where the `construction sites' of the project are.''}& 10.35\\%118\\ 
Code is used by few other methods &--- & 10.96\\%94\\ 	
Simple code was changed\phantom{abcdefgheabcdefghe} &--- & 12.57\\%58\\   %23 characters	43
%Author recently delivered high-quality work	
\bottomrule
\end{tabular}
\label{tab:sub-technical}
%}\vspace{-10pt}
\label{tab:surveyfactors}
%\end{table}
\end{sidewaystable}

\subsubsection{Interviews}
We conducted interviews with ten developers of the RTC team to inform our observational findings. The setting for the interviews was slightly unusual, and beneficial from a research perspective, since our participant-observations allowed us to develop working relationships during our stay at the team, and to delve into a discussion of complex communication issues without needing to spend time understanding the basic context of the team.

In the interviews, we requested developers to provide details into their communication dynamics through a narration of ``war stories'' that directly related to communication among developers. ``\emph{War stories}'' are events that left an impression on the respondent. War stories emphasize specific events that they witnessed or took part~\cite{lutters:ist:2007}; their narrative is particularly powerful at uncovering the elements of the stories that are relevant to the respondents. The interviews lasted between thirty minutes and one hour, and they were conducted at the end of the observation period. 


\subsubsection{Survey}
To complement the insights we obtained from the observations and interviews, we deployed a survey with the entire RTC team at the end of our observation period. The survey was designed iteratively and piloted with a European team, and intended to collect input about which factors increase the likelihood of a developer requesting information about a change-set from the change-set author. The items we included in the survey were in the categories of code-related, developer-related and process-related items. Each category included 14 items, as shown in Table~\ref{tab:surveyfactors}. \emph{Code-change related} items describe the change-set delivered or the code that the change-set modifies or affects. \emph{Developer-related} items relate to
the developers that delivered a change-set, the developer requesting the
information, and their relation to each other. Finally, \emph{process-related} items relate to the development phase in effect when the change-set was delivered, and to items relating to process requirements or practices. For each of the three categories, the survey asked the developers to rank each item in the category according to how strongly it increases their likelihood of requesting information about a change-set. 

We initially formulated our list of items from an analysis of the current literature on coordination and communication, as well as from our own previous research. We then refined the list of items through piloting and discussions with the development team. An initial version of the survey included fifty-nine items; the refined list had forty-two. Besides reducing the number of items in our list, the pilot survey led us to add several process-related items, as well as to change the format of the question (the initial survey had a battery of questions with Likert-scale answers; the final survey asked respondents to rank each item in comparison to the rest in its own category).

We deployed the survey through a web form, and invited the entire distributed RTC development team to participate. We sent a reminder one month later to increase the response rate. We obtained 36 responses to the survey; approximately a 25\% response rate.
Of the 36 respondents, 26 are located in North America, 5 are located in Europe, and 5 are located in Asia. Furthermore, 22 of the respondents are developers, 5 team leads, 5 component leads and project management committee, and 4 respondents withheld their information.

\subsection{Analysis} 
For our data analysis, we transcribed the interviews, and then coded both the interviews and the participant-observation notes. From the codes we created categories, which we cross-referenced to the survey rankings. We used an open coding approach, during which we assigned codes to summarize incidents in sentences and paragraphs of the transcribed interviews that relate to our research question.

We calculated the ranking of the survey items for each category by calculating the mean over all ranks given by developers to a survey item (cf. Table~\ref{tab:surveyfactors}).
Table~\ref{tab:sparkle} shows the distribution of the rankings for each survey item.

During our analysis, we derived a set of factors affecting communication behaviour based on our list of survey items and qualitative data from our interviews and observations. Whenever possible, we triangulated our findings using all our data sources. We tried to ensure that all of our findings were supported by at least two data sources; none of the findings we report were challenged by any of our data sources. 


\section{Findings}
\label{sec:findings}
The evidence collected from our survey, interviews, and participant observation allowed us to derive seven factors that affect communication behaviour in software organizations, which we outline below. In the remainder of this section, text in italics refers to survey items that can be found in Table \ref{tab:surveyfactors}. Text in italics and around quotation marks denotes interview fragments.

\begin{figure}[tb]
\centering
\includegraphics[width=\textwidth]{figures/findingProcess2}
\vspace{-20pt}\caption{The pattern of information-seeking interactions throughout several iterations of a release cycle. Every release cycle consists of a number of iterations; each iteration includes an endgame phase. change-set-based interactions are more frequent during endgame phases and during the last iteration of the release cycle.}
\label{IterationsFig}
\end{figure}



\begin{table}[t!]
\centering
\caption{This table contains the distribution of ranks for each survey item. The leftmost point of each sparkline represents the amount of respondents that ranked the item first; the rightmost point represents the amount that ranked it last (14th).}
\begin{tabular}{rll}
\toprule
%\vspace{-2pt} Process-related items\\
%\midrule
Process-related items &\vspace{-2pt}\includegraphics[height=10px, width=30px]{figures/sparkles/during-release-endgame.pdf} & Release endgame\\
&\vspace{-2pt}\includegraphics[height=10px, width=30px]{figures/sparkles/you-need-to-review-a-change.pdf} & To review the change\\
&\vspace{-2pt}\includegraphics[height=10px, width=30px]{figures/sparkles/you-need-to-approve-a-change.pdf} & To approve the change\\
&\vspace{-2pt}\includegraphics[height=10px, width=30px]{figures/sparkles/late-milestone.pdf} & Late milestone\\
&\vspace{-2pt}\includegraphics[height=10px, width=30px]{figures/sparkles/during-iteration-endgame.pdf} & During iteration endgame\\
&\vspace{-2pt}\includegraphics[height=10px, width=30px]{figures/sparkles/you-need-a-review-for-a-change.pdf} & Obtain a review for the change\\
&\vspace{-2pt}\includegraphics[height=10px, width=30px]{figures/sparkles/you-need-an-approval-for-a-change.pdf} & Obtain an approval for the change\\
&\vspace{-2pt}\includegraphics[height=10px, width=30px]{figures/sparkles/related-work-item-has-high-severity.pdf} & Related work item has high severity\\
&\vspace{-2pt}\includegraphics[height=10px, width=30px]{figures/sparkles/you-need-to-verify-a-fix.pdf} & Verify a fix\\
&\vspace{-2pt}\includegraphics[height=10px, width=30px]{figures/sparkles/topic-of-work-item-the-change-is-attached-to.pdf} & Topic of work item the change is attached to\\ 
&\vspace{-2pt}\includegraphics[height=10px, width=30px]{figures/sparkles/priority-of-related-work-item-was-set-from-your-team-lead.pdf} & Priority set by your team lead\\
&\vspace{-2pt}\includegraphics[height=10px, width=30px]{figures/sparkles/role-of-committer--e-g--developer--team-lead--intern.pdf} & Role of the committer (e.g. developer)\\
&\vspace{-2pt}\includegraphics[height=10px, width=30px]{figures/sparkles/early-in-an-iteration.pdf} & Early in an iteration\\
&\vspace{-2pt}\includegraphics[height=10px, width=30px]{figures/sparkles/early-milestone.pdf} & Early milestone\\
\midrule
%\vspace{-2pt}& Developer-related items\\
%\midrule
Developer-related items&\vspace{-2pt}\includegraphics[height=10px, width=30px]{figures/sparkles/change-author-is-inexperienced.pdf} & Author is inexperienced\\
&\vspace{-2pt}\includegraphics[height=10px, width=30px]{figures/sparkles/recent-work-of-low-quality.pdf} & Author recently delivered sub-standard work\\
&\vspace{-2pt}\includegraphics[height=10px, width=30px]{figures/sparkles/change-author-is-not-up-to-date-with-recent-events.pdf} & Author is not up to date with recent events\\
&\vspace{-2pt}\includegraphics[height=10px, width=30px]{figures/sparkles/don-t-know-the-change-author.pdf} & You do not know the change-set author \\
&\vspace{-2pt}\includegraphics[height=10px, width=30px]{figures/sparkles/currently-working-with-the-change-author.pdf} & Author is currently working with you\\
&\vspace{-2pt}\includegraphics[height=10px, width=30px]{figures/sparkles/change-author-part-of-the-same-feature-team.pdf} & Author part of same feature team\\
&\vspace{-2pt}\includegraphics[height=10px, width=30px]{figures/sparkles/team-of-change-author.pdf} & Author part of your team\\
&\vspace{-2pt}\includegraphics[height=10px, width=30px]{figures/sparkles/worked-with-change-author-in-the-past.pdf} & Worked with author before\\
&\vspace{-2pt}\includegraphics[height=10px, width=30px]{figures/sparkles/busyness-of-yourself} & Busyness of yourself\\
&\vspace{-2pt}\includegraphics[height=10px, width=30px]{figures/sparkles/busyness-of-the-change-author} & Busyness of the author\\
&\vspace{-2pt}\includegraphics[height=10px, width=30px]{figures/sparkles/met-change-author-in-person.pdf} & Met in person\\
&\vspace{-2pt}\includegraphics[height=10px, width=30px]{figures/sparkles/physical-location-of-the-change-author.pdf} & Physical location\\
&\vspace{-2pt}\includegraphics[height=10px, width=30px]{figures/sparkles/change-author-is-experienced.pdf} & Author is experienced\\
&\vspace{-2pt}\includegraphics[height=10px, width=30px]{figures/sparkles/recent-work-of-high-quality.pdf} & Author recently delivered high-quality work\\
\midrule
%\vspace{-2pt}& Code-change related items \\
%\midrule
Code-change related items&\vspace{-2pt}\includegraphics[height=10px, width=30px]{figures/sparkles/change-modified-API.pdf} & Changed API\\
&\vspace{-2pt}\includegraphics[height=10px, width=30px]{figures/sparkles/don-t-know-why-code-was-changed.pdf} & Don't know why code was changed\\
&\vspace{-2pt}\includegraphics[height=10px, width=30px]{figures/sparkles/code-affects-frequently-used-features.pdf} & Affects frequently used features\\
&\vspace{-2pt}\includegraphics[height=10px, width=30px]{figures/sparkles/complex-code-was-changed.pdf} & Complex code\\
&\vspace{-2pt}\includegraphics[height=10px, width=30px]{figures/sparkles/change-introduced-new-functionality.pdf} & Introduced new functionality\\
&\vspace{-2pt}\includegraphics[height=10px, width=30px]{figures/sparkles/code-is-used-by-many-other-methods.pdf} & Is used by many other methods\\
&\vspace{-2pt}\includegraphics[height=10px, width=30px]{figures/sparkles/your-code-was-changed.pdf} & Your code was changed \\
&\vspace{-2pt}\includegraphics[height=10px, width=30px]{figures/sparkles/stable-code-was-changed.pdf} & Stable code was changed\\
&\vspace{-2pt}\includegraphics[height=10px, width=30px]{figures/sparkles/change-unlocks-previously-unused-code.pdf} & Change unlocks previously unused code\\\
&\vspace{-2pt}\includegraphics[height=10px, width=30px]{figures/sparkles/change-was-a-bug-fix.pdf} & A bug fix\\	
&\vspace{-2pt}\includegraphics[height=10px, width=30px]{figures/sparkles/change-was-a-re-factoring.pdf} & A re-factoring\\ 
&\vspace{-2pt}\includegraphics[height=10px, width=30px]{figures/sparkles/frequently-modified-code-was-changed.pdf} & Frequently modified code was changed \\
&\vspace{-2pt}\includegraphics[height=10px, width=30px]{figures/sparkles/code-is-used-by-a-few-other-methods.pdf} & Code is used by few other methods\\ 
&\vspace{-2pt}\includegraphics[height=10px, width=30px]{figures/sparkles/simple-code-was-changed.pdf} & Simple code was changed\\
\bottomrule
\end{tabular}
\vspace{-5pt}
\label{tab:sparkle}
\end{table}


\subsection{Development Mode}
One of the strongest findings from our study was the effect of the development mode that the team is in on its communication behaviour. Developers and managers alike give much importance to the development mode they are in, namely (1) normal iteration, and (2) endgame.

The normal iteration mode mainly consists of work that can be planned by the developer. This work tends to be new feature development, or modifications to existing features. Most of the planning for it has been laid out in advance; furthermore, each developer individually knows what features have been assigned to her, and can plan ahead to meet her obligations.
In contrast, the endgame mode mainly consists of work that is coming uncontrollably in short intervals from others. As a result of integration and more intense testing, defects crop up more rapidly and need to be addressed more quickly. Beyond allocating time for endgame activities, there is very little in this development mode that can be planned in advance.

The RTC team switches between the two modes in each iteration. Of the six weeks of a regular iteration, four weeks are assigned to normal iteration mode and two weeks to endgame. However, as deadlines approach, there are occasional special iterations that consist mostly of endgame-like work. In this manner, the same detailed pattern of alternation between normal iteration and endgame within an iteration is reproduced at a higher level, as shown in Figure~\ref{IterationsFig}.

We identified the same pattern with respect to the amount of change-set-related interactions in the team. We found that the mode in which the team is currently in is an important factor determining whether developers will feel a need to communicate about change-sets. Being in endgame mode increases the need for developers to communicate about a change-set, whereas being in normal mode decreases the need to request information about it.

We experienced both modes during our participant-observer time with the RTC team. We joined the team in a release endgame phase, which was characterized by fixing bugs that were reported by testers, and we helped by fixing minor bugs as well as setting up servers for testing. When the team released the project, it switched gears, and in the decompression after the endgame we had the opportunity to develop a feature for the product. The differences in the patterns of interaction between the two modes were distinctly clear, for us and for our peers. During the endgame mode, developers were essentially ``on demand,'' available to fix whatever bugs were discovered. Later, during the normal iteration mode, they experienced far more autonomy and control over their time, and began working on activities that were less demanding of an interaction back-and-forth, and especially less demanding of keeping track of the change-sets committed by their peers. In our interviews, developers almost unanimously pointed to the separation of the two types of modes, describing the differences in the type of communication in each, and identifying the autonomous \emph{vs.}~uncontrollable, inside \emph{vs.}~outside influence.

This is not to say that developers do not communicate while they are in normal iteration  mode, but that the nature of their communication seems to be different. In normal iteration mode, developers communicate less about concrete change-sets, but they communicate more about high level ideas: for instance, they raise questions about how a feature fits into the existing architecture or general tactics to implement a feature, about how much of it actually needs to be implemented or can be reused from other libraries, and so forth. Generally, however, the amount of communication decreases in normal iteration mode. As a developer commented in one of our interviews, \emph{``during feature development iterations there are weeks I sometimes don't talk to my colleagues at all.''} 

In contrast, the endgame mode is characterized by bug reports and last minute feature requests---that is, by work coming into every developer's desk uncontrollably. change-sets become first class concerns during this mode because each change threatens the stability of the product: developers need to evaluate whether the change-set actually improves the product instead of making things worse. Therefore, they develop a habit of reviewing each other's change-sets to assess the risk they pose to the project's stability.

Our survey data provides significant support for this finding. In Table~\ref{tab:sub-process}, we can see that overall, for the RTC development team at large, there is a greater likelihood to request information during product-stabilizing phases, such as 
\emph{during the release endgame}, and a lesser likelihood to request information when working towards an \emph{early milestone}. The former was ranked as the most important of the process-related items in our survey; the latter was ranked as the least important. All the survey items that refer to a late stage in the iteration or in the project lifecycle were ranked highly, and all the items that refer to an early stage were ranked lowly. This finding resonates especially with the team we interviewed. 


\subsection{Perceived Knowledge of the change-set Author}
One key determinant for a developer to seek information about a change-set consists of what the developer knows about the background of its author. Several aspects of the author's background seem to be important: the quality of her recent work, her level of experience, and her awareness of recent team events or decisions.

First, these factors are important in part because they point to potential problems with the stability of the product. If a change-set author has \emph{recently delivered sub-standard work}, other developers will want to ensure that the new change-set will not deteriorate their product, and therefore they will try to find more information about it.

Similarly, when the author is \emph{inexperienced}, or is not \emph{up to date} with recent team events for any reason, the risk of introducing problems into the product increases. One developer that interacts frequently with a newly founded team told us that \emph{``most of the work that I need to review is of very low quality and needs several iterations before it is up to our standards.''} Novices, generally speaking, face a ramp-up problem that takes a significant amount of time to overcome~\cite{begel:sigcse:2008}. This was the case with our work, too: our first change-sets were subjected to more scrutiny due to our unfamiliarity with the code base.

The survey data backs up these observations. The three top items in Table \ref{tab:sub-social} consist of factors related to an unfavourable perception of the background of the change-set author. In comparison, the two least important items in that table refer to favourable perceptions of said background.

In essence, these items point to the relevance of trust in software development teams. For developers, trust in the skills of their colleagues is important. If this trust is not-established (for instance, if a developer does not know the change-set author), developers are more likely to check the change-set and request information to ensure that the modifications were truly necessary and appropriate. But when trust exists (that is, when the author has a good record of delivering high-quality work, or is known to be an experienced team member), the need to seek information about a change-set decreases. 


\subsection{Common Experience and Location}
Shared work experiences affect the likelihood of information-seeking behaviour in the RTC distributed team. This is partly related to the trust and perception issues discussed above, but also to the establishment of interpersonal relations and to the intertwining work responsibilities and expectations.

According to our interviews, this is the case particularly for senior team members. Although they work in a globally distributed team, they have managed to establish personal relationships with developers at many different sites. Through continuous interaction and through social events, they have learned more about each other, and thus feel more comfortable to initiate contact with them. Junior team members, in contrast, know very few of their teammates globally, and since they simply do not have these interpersonal connections with the rest of their team, it is less important to them whether they know the author of a change-set when needing to contact them.

Generally speaking, despite the lack of opportunities to build personal relationships with off-site people, for most developers sharing common experiences makes it easier to contact teammates to request information. One team lead told us: \emph{``I have one or two contact people in each team we usually work with whom I ask for information and then often enough get referred to the right person in their team.''}

Our participant-observation data confirm this. At one point, we needed to set up and test a specific component in the RTC product. The development team in charge of that component was located on a different site. However, since we previously had the opportunity to establish relationships with some of the developers of that team through a previous research study, it was easier for us to contact them and ask for help in our setup task.

Our survey data corroborate these observations. It suggests that the developers that form relationships (on a personal or work level) with other developers gave more emphasis on previous experiences with them. For instance, it made a difference whether the developers \emph{are currently working together} (cf. Table~\ref{tab:sub-social}). Additionally, whether they have shared work experiences seems to be a more determinant factor than having \emph{met in person} or sharing the same \emph{physical location}.

This last finding would seem to indicate that the RTC team has managed to overcome obstacles created by geographical distance. Developers state that the physical location of a change-set author is not an important factor to seek more information about the change-set. Nevertheless, we should note two things. First, this finding merely points out that developers have no greater need to inform themselves of work output of their remote peers than of their local peers; it says nothing about the ease with which such information is acquired. Second, the RTC team has determined to carry out as many of its interactions as possible through textual, electronic media. This neglects the natural advantages provided by proximity in favour of uniform accessibility of interactions, no matter the physical location of the interlocutors.



\subsection{Risk Assessment}
A central factor at the heart of the decision to seek information is a concern with the quality of the product and components developed by the team at large. Both managers and developers share this concern. Because of it, every team member is constantly evaluating whether there are significant risks involved in accepting a change-set and including it into the final product. This concern is greatest close to releases or major milestones, and less important when the team is in normal iteration mode.

Developers request information more frequently about change-sets that touch on code that has a high customer impact. According to our interviews and observations, code parts with a high customer impact are those that are directly related to frequently used features or changes to the API that might be used by customers to customize the RTC product. In the case of the impact of API changes, there is an extensive knowledge exchange in the jazz.net forums between customers and developers about how to use the API to build extensions to RTC, which gives developers plenty of information about the possible impact of API changes to the customers. Consequently, change-sets that modify code that has less impact on the customer are of less concern and less likely to be discussed. For example, when scanning fixes to reported bugs, developers perform a risk assessment to determine whether the bug fix is even needed. In the words of one team member, \emph{``with every tenth bug fix you introduce another bug to the system, so unless the bug does not have a workaround and is something most customers would experience, we give it low priority.''}

The ranking of items in Table \ref{tab:sub-technical} provides additional insights into the kinds of change-sets that carry a risk for developers. A \emph{changed API} comes at the top of the list, code that \emph{affects frequently used features} comes third, immediately followed by changes to \emph{complex code}, changes that \emph{introduced new functionality}, and changes to code that \emph{is used by many other methods}.


\subsection{Work Allocation and Peer Reviews}
In the later stages of the development cycle, developers have two process-based constraints that prompt them to interact with their peers. First, they need their peers to review their change-sets before they are included in the product. Second, they need an approval to start working on a task (a triaging process needs to take place so that developers are allocated to the most important work items).

In the change-set review process, there are two ways the change-set author and the reviewers interact.
Either the change-set author submits his change-set for review, or the change-set author discusses the change-set with the reviewer before submitting it for review. The main discriminant between meeting with the reviewer to discuss the change-set before submitting it or submitting without prior discussion seems to be whether the reviewer and the author are co-located.

Our mentor during our time with the RTC team explained: \emph{``It is very demotivating to get a change-set rejected, that's why we usually try to discuss things before and then the reviewing just becomes all about quickly testing and accepting the change-set.''} Of course, this does not prevent an already discussed change-set from being rejected, but it signals that the process-based constraint is not as strict as to forbid discussions about a change-set between authors and reviewers in the interest of an unbiased code review.

In the case of developers seeking approval to start working on a task, the process is similarly flexible. The approval is supposed to be issued before the developer begins work on the task. However, we observed several times that developers performed quick fixes, and later went to their team leads to acquire approval for the fixes they had already performed.

Several items in our survey point to the importance of information-seeking around the processes of approval and peer review. In Table \ref{tab:sub-process}, the second- and third-highest items were information needed because the respondent needed \emph{to review the change} or \emph{to approve the change}. The need to  \emph{obtain a review for the change} or to \emph{obtain an approval for the change} were of moderate importance. \emph{Verifying a fix} was not judged as important by our respondents.

\subsection{Type of Change}
The three main reasons for a developer to commit a change-set are to (1) do feature development, (2) fix bugs, and (3) re-factor. Among them, feature development seems to be the one that prompts developers to seek information the most.
In our survey (cf. Table \ref{tab:sub-process}), code that \emph{introduced new functionality} is ranked fifth among the code-change related factors, while changes that consist of \emph{a bug fix} and \emph{a refactoring} ranked far lower on the same list.

This may be due to the fact that feature development has less tool support than bug fixing (which is supported by debugging tools and automated test frameworks) and refactoring (which is essentially an automatic process today). The limited tool support and the inherently difficult nature of the task itself makes change-sets that introduce new functionality more difficult to understand or assess. Hence, developers are more likely to request information about them.


\subsection{Business Goals vs Developer Pride}
Senior team members (managers and team leads among them) often ask about the purpose and the real need of a change-set in order to minimize unpleasant surprises that might be introduced with further code changes, and simply to have an economically feasible product. One senior development lead told us: \emph{``often I need to stop my developers from fixing every little bug otherwise we would never be able to ship.''}

Although every team member is concerned with the quality of their product, a more realistic, business-oriented perspective seems to be more prominent in the senior team members' information-seeking behaviour: they may request clarifications on the business case of change-sets or work items if they are not immediately clear in the existing documentation.

More junior team members, however, do not have this concern. They express their pride in the actual product and want it to be as bug free as possible. A developer commented to us: \emph{``I want to be proud of what I deliver and I am fairly certain if we don't fix it now it will come back to haunt us because it might upset some customers.''} Not only does this lead the developer to try to convince her manager that including change-sets for seemingly unimportant fixes is valuable, it also makes her information-seeking behaviour different with respect to that of the more senior team members. Broadly speaking, managers will be more interested in seeking clarifications as to the business case of a work item if it is not clear in its description, while developers will not care as much about this.


\section{Summary}
\label{sec:conclusions}
We end this chapter by bringing it back to the initial research question we set out to answer:
\begin{description}
  \item[RQ 2.2:] Do developers accept recommendations based on software changes to increase build success? 
\end{description}

The findings we presented in Section~\ref{sec:findings} show that there are different factors that influence developers to inquire about change-sets.
Although not directly asked for, the developer who reported on those factors answered our research question in a positive way.
Our approach would provide developers with recommendations either on a per change-set level or while they are working on the code.
We found that developers do talk about change-sets which represent our level of recommendations as well as they show interest in build outcome towards the end of a release cycle.
Thus we can say that the level of our recommendation seems to be appropriate when keeping it to the change-set level.
Examining our finding about risk assessment, which is practiced by every lead and developer we are confident that developers would welcome notifications on how risky with respect to build success the changes they are currently applying are to give them a chance to either talk to the recommended person or reconsider the change altogether.
Hence, we are confident the developer answers both validated that they would accept recommendations both in the form we present them, namely per change-set submitted, as well as lending our outcome metric more credibility as they show a keen interest in build success when the date to ship the product approaches.

In the next chapter we return to our path of a more direct analysis of our recommendations by showing a proof of concept that the recommendations do not only statistically relate to build outcome but that recommendation could actually prevent build failures.