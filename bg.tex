\startchapter{Background}
In this chapter we will give a rough overview over the challenge that we described in the introduction and how other work tried partially or in full to address these challenges.
Note that we will not go into great detail of the indvidual steps that we undertook in this thesis as we discuss related work in the appropriate chapter.
Thus we will discuss two areas: (1) the research around the concept of socio-technical congruence and (2) recommender systems in sofftware engineering.

\section{Socio-Technical Congruence}
As mentioned earlier this thesis explores to what extend cane we leverage the concept of socio-technical congruence, before we start discussion the work that has been done with respect to using the concept of socio-technical congruence to analyise softare development teams and their performance, we explain the socio-technical congruence conpect and it different ways it was implemented so far.
We close this section by discussion the next step we are going to take with respect to concept.

\subsection{Socio-Technical Congruence Definitions}
The literature ecploring and using the concept of socio-technical congruence often relies on two interconnected definitions of socio-technical congruence.
Originally as first defined by Cataldo et al~\cite{} socio-techncial congruence was a single metric describing how much of the work dependencies between developersa are covered by the commonuication between those developers, but the interest in socio-technical congruence took a broader view and instead of focusing on the metric researchers~\cite{} started focusing on the underlying construct conceptualizing the different connections among dvelopers.
Following we discuss the two commonly used approached to infer socio-technical dependecies among developers, starting with the traditional definition initially presented by Cataldo et al~\cite{} followed by a more network centric definition.

\subsubsection{Task Assignment and Dependecy}
\subsubsection{Social and Technical Networks}

\subsection{Socio-Technical Congruence and Performance}
\subsection{Socio-Technical Congruence Next Steps}

\section{Recommender Systems in Software Engineering}
